\documentclass[12pt]{article}

\usepackage{fullpage}
\usepackage{multicol,multirow}
\usepackage{tabularx}
\usepackage{ulem}
\usepackage[utf8]{inputenc}
\usepackage[russian]{babel}
\usepackage{minted}

\usepackage{color} %% это для отображения цвета в коде
\usepackage{listings} %% собственно, это и есть пакет listings

\lstset{ %
language=C,                 % выбор языка для подсветки (здесь это С)
basicstyle=\small\sffamily, % размер и начертание шрифта для подсветки кода
numbers=left,               % где поставить нумерацию строк (слева\справа)
%numberstyle=\tiny,           % размер шрифта для номеров строк
stepnumber=1,                   % размер шага между двумя номерами строк
numbersep=5pt,                % как далеко отстоят номера строк от подсвечиваемого кода
backgroundcolor=\color{white}, % цвет фона подсветки - используем \usepackage{color}
showspaces=false,            % показывать или нет пробелы специальными отступами
showstringspaces=false,      % показывать или нет пробелы в строках
showtabs=false,             % показывать или нет табуляцию в строках
frame=single,              % рисовать рамку вокруг кода
tabsize=2,                 % размер табуляции по умолчанию равен 2 пробелам
captionpos=t,              % позиция заголовка вверху [t] или внизу [b] 
breaklines=true,           % автоматически переносить строки (да\нет)
breakatwhitespace=false, % переносить строки только если есть пробел
escapeinside={\%*}{*)}   % если нужно добавить комментарии в коде
}


\begin{document}
\begin{titlepage}
\begin{center}
\textbf{МИНИСТЕРСТВО ОБРАЗОВАНИЯ И НАУКИ РОССИЙСОЙ ФЕДЕРАЦИИ
\medskip
МОСКОВСКИЙ АВЦИАЦИОННЫЙ ИНСТИТУТ
(НАЦИОНАЛЬНЫЙ ИССЛЕДОВАТЬЕЛЬСКИЙ УНИВЕРСТИТЕТ)
\vfill\vfill
{\Huge ЛАБОРАТОРНАЯ РАБОТА №8} \\
по курсу объектно-ориентированное программирование
I семестр, 2019/20 уч. год}
\end{center}
\vfill

Студент \uline{\it {Попов Данила Андреевич, группа М8О-208Б-18}\hfill}

Преподаватель \uline{\it {Журавлёв Андрей Андреевич}\hfill}

\vfill
\end{titlepage}

\subsection*{Условие}

Работа с ассинхронностью.

Редактор должен соответствовать следующему функционалу:
\begin{enumerate}
\item размер буфера должен задаваться через командную строку
\item результат обработки буфера должен выводиться на экран и в файл
\item в программе должно быть два потока
\item должен прослеживаться паттерн publish-subscribe
\end{enumerate}

\subsection*{Описание программы}

Исходный код лежит в 12 файлах:
\begin{enumerate}
\item app/main.cpp
\item include/async.hpp
\item include/point.hpp
\item include/polygon.hpp
\item include/publisher.hpp
\item include/serializable.hpp
\item include/subscriber.hpp
\item src/async.cpp
\item src/publisher.cpp
\item src/serializable.cpp
\item src/subscriber.cpp
\end{enumerate}

\subsection*{Дневник отладки}

Race condition при инициализации второго потока.

\subsection*{Недочёты}

На одну структуру приходится два вложенных shared_ptr (смотреть my_event).

\subsection*{Выводы}

Странная лабораторная работа с мультипоточностью, в которой одновременно может выполняться только один поток.

\vfill

\subsection*{Исходный код}

{\Huge main.cpp}
\inputminted
    {C++}{app/main.cpp}
    \pagebreak

{\Huge include/async.hpp}
\inputminted
    {C++}{include/async.hpp}
    \pagebreak

{\Huge src/async.cpp}
\inputminted
    {C++}{src/async.cpp}
    \pagebreak

{\Huge include/point.hpp}
\inputminted
    {C++}{include/point.hpp}
    \pagebreak

{\Huge include/polygon.hpp}
\inputminted
    {C++}{include/polygon.hpp}
    \pagebreak

{\Huge include/publisher.hpp}
\inputminted
    {C++}{include/publisher.hpp}
    \pagebreak

{\Huge src/publisher.cpp}
\inputminted
    {C++}{src/publisher.pp}
    \pagebreak

{\Huge include/serializable.hpp}
\inputminted
    {C++}{include/serializable.hpp}
    \pagebreak

{\Huge src/serializable.cpp}
\inputminted
    {C++}{src/serializable.cpp}
    \pagebreak

{\Huge include/subscriber.hpp}
\inputminted
    {C++}{include/subscriber.hpp}
    \pagebreak

{\Huge src/subscriber.cpp}
\inputminted
    {C++}{src/subscriber.cpp}
    \pagebreak
    
\end{document}
